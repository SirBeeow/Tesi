\chapter{Svolgimento \emph{stage}}
\label{cap:svolgimentoStage}
\section{Analisi}
\subsection{Tecnologie oggetto di \emph{stage}}
Questa sottosezione descirve i metodi di apprendimento e le nozioni apprese durante la fase di analisi delle tecnologie su cui si basa la ricerca del mio progetto di \emph{stage}.\\
Dopo aver compreso le tecnologie e i principali processi aziendali, ho partecipato ad un \emph{meeting} con il \emph{tutor} aziendale al fine di discutere il mio progetto di \emph{stage}.\\
Gli obiettivi scaturiti da tale incontro sono stati: 
\begin{itemize}
    \item Autoapprendimento dello strumento Power Automate.
    \item Realizzazione di un PoC che testasse la possibilità di realizzare un flusso approvativo Power Automate. 
    \item Testare le funzionalità disponibili con le licenze di utilizzo \emph{standard}. 
\end{itemize}

\subsection{Attività sulle applicazioni aziendali}
\label{sviluppoApplicazioni}




