\chapter{Valutazione retrospettiva}
\label{cap:valutazioneRetrospettiva}
\section{Raggiungimento obiettivi}
In riferimento alla \hyperref[tab:obiettiviProgettuali]{tabella degli obiettivi progettuali}, in seguito vengono riportate le azioni da me intraprese al fine di raggiungere il loro soddisfacimento e i corrispondenti esiti.

\begingroup
\renewcommand\arraystretch{1.3}
\begin{longtable}{|c|p{8cm}|c|}
    \caption{Soddisfacimento degli obiettivi obbligatori.}
    \label{tab:soddObbObbligatori}\\
    \hline \textbf{Requisito} & \textbf{Azioni risolutive} & \textbf{Esito}\\ \endfirsthead
    \hline \textbf{Requisito} & \textbf{Azioni risolutive} & \textbf{Esito}\\ \endhead
    \hline \endfoot
    \hline \endlastfoot
    \hline O1.1  & A seguito dell'analisi svolta sui flussi Power Automate, ho compreso le informazioni relative alle sue funzionalità, caratteristiche, punti positivi e punti negativi. Ho redatto il documento “Presentazione Power Automate e Power Apps” al fine di esporre il risultato di tali ricerche. & Soddisfatto\\
    \hline O1.2  & Ho prodotto molteplici flussi Power Automate: flussi per il \emph{retrive} dei dati da SharePoint, flussi di approvazione e per l'integrazione con chiamate \gls{http}. & Soddisfatto\\
    \hline O1.3  & Il documento “Presentazione Power Automate e Power Apps” è stato esposto e discusso con il \emph{tutor} aziendale e con il \emph{team} di sviluppo. & Soddisfatto\\
    \hline \textbf{O1}    & Ho studiato, analizzato ed esplorato la tecnologia Power Automate, redigendo i relativi documenti, i quali ho poi condiviso con il \emph{team} aziendale. & Soddisfatto\\
    \hline O2.1  & Ho affiancato il membro del \emph{team} di sviluppo responsabile della realizzazione dei prodotti Power Apps aziendali. Egli mi ha spiegato dettagliatamente il funzionamento del programma e delle applicazioni create. & Soddisfatto\\
    \hline O2.2  & Ho individuato, come sistema per lo sviluppo collaborativo di applicazioni Power Apps, la condivisione del materiale su Git mediante l'apposito comando integrato. Ho redatto il documento “Norme di versionamento” al fine di normare e standardizzare il suo utilizzo e le strategie di collaborazione. & Soddisfatto\\
    \hline O2.3  & Ho sviluppato, collaborativamente e autonomamente, flussi Power Automate e formule Power Fx, ottenendo avanzamenti nello stato dei lavori di applicazioni aziendali realizzate con Power Apps. & Soddisfatto\\
    \hline \textbf{O2}  & Ho studiato Power Apps mediante ricerca individuale e formazione collaborativa. Ho ottenuto avanzamenti nello stato dei lavori su applicazioni aziendali realizzate con tale tecnologia. & Soddisfatto\\
    \hline \textbf{O3}  & Ho individuato, analizzato, testato ed applicato un sistema per il versionamento di progetti realizzati con Power Automate e Power Apps basato su \emph{repository} Git. Le norme relative al suo utilizzo sono state approvate dal \emph{team} di sviluppo e sono state da me descritte approfonditamente nel documento “Norme di versionamento”. & Soddisfatto\\
    \hline \textbf{O4}  & Ho studiato la metodologia \gls{DevOps} e tutte le sue fasi. Ho studiato le possibilità di applicarle alle tecnologie Power Automate e Power Apps, individuando soluzioni specifiche che ho poi sviluppato e integrato ai relativi progetti aziendali. Al fine di condividere con il \emph{team} tali informazioni, ho redatto i documenti “Norme di versionamento”, “Analisi statica del codice” e “Test dinamici”. Inoltre, al fine di descrivere la tecnologia utilizzata per gestire la maggior parte di queste fasi, ho redatto il documento “Guida Jenkins”. & Soddisfatto\\
\end{longtable}

\begin{longtable}{|c|p{8cm}|c|}
    \caption{Soddisfacimento degli obiettivi desiderabili.}
    \label{tab:soddObbDesiderabili}\\
    \hline \textbf{Requisito} & \textbf{Azioni risolutive} & \textbf{Esito}\\ \endfirsthead
    \hline \textbf{Requisito} & \textbf{Azioni risolutive} & \textbf{Esito}\\ \endhead
    \hline \endfoot
    \hline \endlastfoot
    \hline \textbf{D1}  & Ho collaborato con gli altri due stagisti universitari al fine di condividere reciprocamente, mediante appositi \emph{meeting} e condivisione dei documenti prodotti, le attività svolte e le nozioni apprese durante i nostri \emph{stage}. Ho con loro collaborato al fine di individuare una soluzione che sincronizzasse gli strumenti Planner e Taiga mediante flussi Power Automate. Inoltre abbiamo discusso e definito parte delle strategie da adottare relativamente al sistema di versionamento per progetti Power Automate e Power Apps. & Soddisfatto\\
    \hline D2.1  & Ho realizzato un flusso Power Automate al fine di dimostrare la fattibilità tecnologica di flussi in grado di inviare richieste di approvazione e reagire contestualmente alla risposta ricevuta. & Soddisfatto\\
    \hline D2.2  & Ho realizzato un flusso Power Automate per dimostrare la fattibilità tecnologica di flussi in grado di ricevere e inviare chiamate \gls{http}, al fine di scambiare dati con \emph{script} e servizi esterni. & Soddisfatto\\
    \hline D2.3  & Ho realizzato un Multibranch \emph{pipeline} Job Jenkins, e relativo Jenkinsfile, al fine di dimostrare l'applicabilità della fase \emph{Build} di \gls{DevOps} a progetti realizzati con Power Automate e Power Apps. Esso include uno \emph{Stage} responsabile per l'esportazione dei prodotti \emph{software}, sotto forma di pacchetti, e il loro caricamento nel \emph{repository} Git. & Soddisfatto\\
    \hline D2.4  & Ho realizzato un Multibranch \emph{pipeline} Job Jenkins, e relativo Jenkinsfile, al fine di dimostrare l'applicabilità della fase \emph{Test} di \gls{DevOps} a progetti realizzati con Power Automate e Power Apps. Esso include uno \emph{Stage} responsabile per l'esecuzione di uno \emph{script} il quale, tramite chiamate \gls{http}, esegue un flusso Power Automate ottenendone l'\emph{output}. Quest'ultimo viene poi automaticamente analizzato al fine di testarne la correttezza. & Soddisfatto\\
    \hline \textbf{D2}  & Ho testato le soluzioni individuate durante l'analisi e la progettazione mediante dimostrazioni e PoC. Ho poi condiviso con il \emph{team} i documenti relativi ai risultati ottenuti, i quali comprendono i PoC sull'utilizzo di Jenkins e alle \emph{features} disponibili con l'utilizzo di flussi Power Automate. & Soddisfatto\\
    \hline \textbf{D3}  & Ho collaborato con lo stagista universitario incaricato di analizzare l'applicabilità delle pratiche \gls{DevOps} in ambito \gls{Sistemi}, al fine di comprenderne le soluzioni individuate, i limiti tecnologici e i benefici guadagnati. Questo è avvenuto tramite appositi \emph{meeting} e condivisione dei documenti redatti. & Soddisfatto\\
\end{longtable}

\begin{longtable}{|c|p{8cm}|c|}
    \caption{Soddisfacimento degli obiettivi facoltativi.}
    \label{tab:soddObbFacoltativi}\\
    \hline \textbf{Obiettivo} & \textbf{Azioni risolutive} & \textbf{Esito}\\  \endfirsthead
    \hline \textbf{Obiettivo} & \textbf{Azioni risolutive} & \textbf{Esito}\\  \endhead
    \hline \endfoot
    \hline \endlastfoot
    \hline \textbf{F1}  & Ho studiato ed esplorato lo strumento Angular, e con esso ho realizzato un'applicazione \emph{web} contenente testo e immagini. & Soddisfatto\\
    \hline \textbf{F2}  & Ho applicato le principali fasi di \gls{DevOps} al progetto Angular da me realizzato, mediante l'utilizzo di un \emph{pipeline} Job Jenkins. & Soddisfatto\\
    \hline \textbf{F3}  & Ho realizzato, in collaborazione con gli altri stagisti universitari, una presentazione finale contenente tutti i risultati raggiunti durante lo \emph{stage}. Essa è stata poi esposta al presidente di Wintech e ai responsabili. & Soddisfatto\\
\end{longtable}
\endgroup

\section{Maturazione professionale}
Nel periodo di \emph{stage} ho affrontato diverse sfide a me nuove in ambito lavorativo, le quali hanno permesso una mia crescita a livello professionale e l'acquisizione di nuove competenze.
Tra queste è compresa la capacità di compiere analisi approfondite riguardo a tecnologie non precedentemente affrontate, l'individuazione degli strumenti da adottare all'interno di un progetto e l'autoapprendimento relativo al loro utilizzo.\\
Ho maturato la capacità di produrre documentazione professionale, conforme agli \emph{standard} aziendali, atta a condividere i risultati delle mie ricerche e a normare processi lavorativi.\\ 
Ho sviluppato competenze relative alla collaborazione con i responsabili e con i \emph{team} aziendali organizzando \emph{meeting}, utilizzando correttamente le tecnologie designate e tramite le fasi di progettazione e sviluppo cooperativo.\\
Ho acquisito ulteriore esperienza relativamente all'autogestione delle attività, al fine di raggiungere gli obiettivi previsti nei tempi e con gli strumenti prestabiliti, mantenendo il rispetto per la qualità attesa e i processi aziendali definiti.\\
Infine, ho sviluppato le mie capacità espositive e comunicative eseguendo presentazioni strutturate e professionali, al fine di condividere le informazioni relative al lavoro da me svolto.\\\\
Ritengo che questa sia stata un'esperienza assolutamente positiva, durante la quale ho collaborato con persone disponibili e propositive, acquisendo nuove competenze e guadagnando esperienza. 

\section{Divario formativo}
Durante lo svolgimento dello \emph{stage}, ho potuto constatare come la formazione universitaria da me intrapresa mi abbia preparato ad affrontare diversi argomenti incontrati nell'ambiente lavorativo.\\
Tra questi sono comprese le filosofie Agile e Scrum, la gestione del ciclo di vita del \emph{software}, le metodologie di automazione \gls{Continuous Integration} e \gls{Continuous Deployment}, e lo scopo delle fasi di \gls{DevOps}.\\
Inoltre ritengo che il progetto universitario compreso nell'insegnamento "Ingegneria del Software", sia stato fondamentale per comprendere come affrontare un progetto, enfatizzando l'importanza dell'analisi, progettazione e redazione dei relativi documenti normativi.\\
Per quanto concerne l'ambito tecnologico, avevo precedentemente affrontato i temi riguardanti le tecnologie Git e Jenkins, e avevo necessariamente già utilizzato IDE e la maggior parte dei linguaggi di programmazione utilizzati durante lo \emph{stage}.\\ 
In ambito informatico esistono una moltitudine di differenti tecnologie, le quali vengono frequentemente aggiornate e modificate. 
Per questo motivo, il corso universitario mi ha formato al fine di apprendere efficacemente nuovi strumenti e tecnologie in autonomia. 
