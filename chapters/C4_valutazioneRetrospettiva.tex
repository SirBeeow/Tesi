\chapter{Valutazione retrospettiva}
\label{cap:valutazioneRetrospettiva}

\section{Raggiungimento obiettivi}
\begin{longtable}{|c|p{7cm}|c|}
    \caption{Soddisfacimento degli obiettivi obbligatori.}
    \label{tab:soddObbObbligatori}\\
    \hline \textbf{Requisito} & \textbf{Azioni risolutive} & \textbf{Soddisfacimento}\\ \endfirsthead
    \hline \textbf{Requisito} & \textbf{Azioni risolutive} & \textbf{Soddisfacimento}\\ \endhead
    \hline \endfoot
    \hline \endlastfoot
    \hline O1.1  & A seguito dell'analisi svolta sui flussi Power Automate, ho compreso le informazioni relative alle sue funzionalità, caratteristiche, punti positivi e punti negativi. Ho redatto il documento “Presentazione Power Automate e Power Apps” al fine di esporre il risultato di tali ricerche. & Soddisfatto\\
    \hline O1.2  & Ho prodotto multipli flussi Power Automate: flussi per il \emph{retrive} dei dati da SharePoint, flussi di approvazione e per l'integrazione con chiamate \gls{http}. & Soddisfatto\\
    \hline O1.3  & Il documento “Presentazione Power Automate e Power Apps” è stato esposto e discusso con il \emph{tutor} aziendale e con il \emph{team} di sviluppo. & Soddisfatto\\
    \hline \textbf{O1}    & Ho studiato, analizzato ed esplorato la tecnologia Power Automate, redigendo i relativi documenti, i quali ho poi condiviso con il \emph{team} aziendale. & Soddisfatto\\
    \hline O2.1  & Ho affiancato il membro del \emph{team} di sviluppo responsabile della realizzazione dei prodotti Power Apps aziendali. Egli mi ha spiegato dettagliatamente il funzionamento del programma e delle applicazioni create. & Soddisfatto\\
    \hline O2.2  & Ho individuato, come sistema per lo sviluppo collaborativo di applicazioni Power Apps, la condivisione del materiale su Git mediante l'apposito comando integrato. Ho redatto il documento “Norme di versionamento” al fine di normare e standardizzare il suo utilizzo e le strategie di collaborazione. & Soddisfatto\\
    \hline O2.3  & Ho sviluppato, collaborativamente e autonomamente, flussi Power Automate e formule Power Fx, ottenendo avanzamenti nello stato dei lavori di applicazioni aziendali realizzate con Power Apps. & Soddisfatto\\
    \hline \textbf{O2}  & Ho studiato Power Apps mediante ricerca individuale e formazione collaborativa. Ho ottenuto avanzamenti nello stato dei lavori su applicazioni aziendali realizzate con tale tecnologia. & Soddisfatto\\
    \hline \textbf{O3}  & Ho individuato, analizzato, testato ed applicato un sistema per il versionamento di progetti realizzati con Power Automate e Power Apps basato su \emph{repository} Git. Le norme relative al suo utilizzo sono state approvate dal \emph{team} di sviluppo e sono state da me descritte approfonditamente nel documento “Norme di versionamento”. & Soddisfatto\\
    \hline \textbf{O4}  & Ho studiato la metodologia \gls{DevOps} e tutte le sue fasi. Ho studiato le possibilità di applicarle alle tecnologie Power Automate e Power Apps, individuando soluzioni specifiche che ho poi sviluppato e integrato ai relativi progetti aziendali. Al fine di condividere con il \emph{team} tali informazioni, ho redatto i documenti “Norme di versionamento”, “Analisi statica del codice” e “Test dinamici”. Inoltre, al fine di descrivere la tecnologia utilizzata per gestire la maggior parte di queste fasi, ho redatto il documento “Guida Jenkins”. & Soddisfatto\\
\end{longtable}

\begin{longtable}{|c|p{7cm}|c|}
    \caption{Soddisfacimento degli obiettivi desiderabili.}
    \label{tab:soddObbDesiderabili}\\
    \hline \textbf{Requisito} & \textbf{Azioni risolutive} & \textbf{Soddisfacimento}\\ \endfirsthead
    \hline \textbf{Requisito} & \textbf{Azioni risolutive} & \textbf{Soddisfacimento}\\ \endhead
    \hline \endfoot
    \hline \endlastfoot
    \textbf{D1}  & & Soddisfatto\\
    \hline D2.1  & & Soddisfatto\\
    \hline D2.2  & & Soddisfatto\\
    \hline D2.3  & & Soddisfatto\\
    \hline D2.4  & & Soddisfatto\\
    \hline \textbf{D2}  & & Soddisfatto\\
    \hline \textbf{D3}  & & Soddisfatto\\
\end{longtable}

\begin{longtable}{|c|p{7cm}|c|}
    \caption{Soddisfacimento degli obiettivi facoltativi.}
    \label{tab:soddObbFacoltativi}\\
    \hline \textbf{Obiettivo} & \textbf{Azioni risolutive} & \textbf{Soddisfacimento}\\  \endfirsthead
    \hline \textbf{Obiettivo} & \textbf{Azioni risolutive} & \textbf{Soddisfacimento}\\  \endhead
    \hline \endfoot
    \hline \endlastfoot
    \textbf{F1}  & & Soddisfatto\\
    \hline \textbf{F2}  & & Soddisfatto\\
    \hline \textbf{F3}  & & Soddisfatto\\
\end{longtable}

\section{Maturazione professionale}
%

\section{Divario formativo}
%

