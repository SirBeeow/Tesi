\chapter{Svolgimento stage}
\label{cap:svolgimentoStage}
In questo capitolo vengono descritte tutte le attività da me svolte durante lo \emph{stage}, divise nelle sezioni “Analisi”, “Progettazione”, “Programmazione” e “Verifica e validazione”, in modo da fornire una panoramica chiara e strutturata del lavoro svolto, evidenziando il processo seguito e le competenze acquisite in ciascuna fase.\\
Esse non sono intese come completamente sequenziali bensì, fin dalla prima fase, sono presenti tutte le attività correlate in modo da coprire l'intero periodo di \emph{stage}.\\
\section{Analisi}
In questa sezione sono presenti tutte le attività analitiche da me svolte e il suo scopo è descrivere le modalità con cui ho compreso i bisogni e i requisiti del mio progetto di \emph{stage}. 

\subsection*{Requisit}
Gli obiettivi del mio \emph{stage}, come dichiarati nel documento “Progetto Formativo” generato all'inizio del suo svolgimento, sono divisi in categorie:
\begin{enumerate}
	\item[O -]requisiti obbligatori, vincolanti in quanto obiettivi primari richiesti dall'azienda.
    \item[D -]requisiti desiderabili, non strettamente necessari ma dal riconoscibile valore aggiunto.
    \item[F -]requisiti facoltativi / opzionali, rappresentanti un valore aggiunto non strettamente competitivo.\\
\end{enumerate}
Essi sono:

\begin{table}[htbp]
    \label{tab:obiettiviProgettuali}
    \renewcommand{\arraystretch}{1.5}
    \begin{tabularx}{\textwidth}{|l|X|l|}
    \hline
    \textbf{Codice} & \textbf{Descrizione}\\
    \hline
    O1    & Mappatura delle funzionalità possibili tramite l'adozione dei due applicativi \gls{Sistemi} e Office365.\\
    \hline O1.1  & Analisi approfondita del sistema e delle parti interessate.\\
    \hline O1.2  & Studio delle modalità di lavoro degli utenti.\\
    \hline O1.3  & Produzione di una completa documentazione di uso.\\
    \hline O2  & Personalizzazione e integrazione: individuare le modalità di utilizzo.\\
    \hline
    \hline D1  & Analisi dei requisiti per l'integrazione aziendale della metodologia \gls{DevOps} in ambito \GLS{Sistemi} e Office365.\\
    \hline D2  & Produzione di una completa documentazione progettuale.\\
    \hline
    \hline F1  & Realizzazione di \emph{Proof of concept}.\\
    \hline F2  & Presentazione interna.\\
    \hline F3  & Predisposizione della documentazione.\\
    \hline
    \end{tabularx}
    \caption{Tabella degli obiettivi progettuali.}
\end{table}%
\subsection*{Ambiente di lavoro}


\subsection*{Requisiti progettuali}


\section{Progettazione}
In questa sezione sono presenti tutte le attività progettuali da me svolte e Il suo scopo è descrivere le modalità con cui ho individuato le soluzioni ai bisogni progettuali in modo da soddisfarne i requisiti.

\section{Programmazione}
In questa sezione sono presenti tutte le attività da me svolte al fine di sviluppare e implementare le soluzioni individuate in fase di progettazione.

\section{Verifica e Validazione}
In questa sezione sono presenti tutte le attività da me svolte al fine di verificare il corretto funzionamento delle soluzioni sviluppate e il loro soddisfacimento dei requisiti progettuali.

\section{Risultati raggiunti}
\subsection{Qualitativamente}
%

\subsection{Quantitativamente}
%
