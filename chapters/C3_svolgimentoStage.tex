\chapter{Svolgimento stage}
\label{cap:svolgimentoStage}
In questo capitolo vengono descritte tutte le attività da me svolte durante lo \emph{stage}, divise nelle sezioni “Analisi”, “Progettazione”, “Programmazione” e “Verifica e validazione”, in modo da fornire una panoramica chiara e strutturata del lavoro svolto, evidenziando il processo seguito e le competenze acquisite in ciascuna fase.\\
Esse non sono intese come completamente sequenziali bensì, fin dalla prima fase, sono presenti tutte le attività correlate in modo da coprire l'intero periodo di \emph{stage}.\\
\section{Analisi}
In questa sezione sono presenti tutte le attività analitiche da me svolte e il suo scopo è descrivere le modalità con cui ho compreso i bisogni e i requisiti del mio progetto di \emph{stage}. 

\section{Progettazione}
In questa sezione sono presenti tutte le attività progettuali da me svolte e Il suo scopo è descrivere le modalità con cui ho individuato le soluzioni ai bisogni progettuali in modo da soddisfarne i requisiti.

\section{Programmazione}
In questa sezione sono presenti tutte le attività da me svolte al fine di sviluppare e implementare le soluzioni individuate in fase di progettazione.

\section{Verifica e Validazione}
In questa sezione sono presenti tutte le attività da me svolte al fine di verificare il corretto funzionamento delle soluzioni sviluppate e il loro soddisfacimento dei requisiti progettuali.

\section{Risultati raggiunti}
\subsection{Qualitativamente}
%

\subsection{Quantitativamente}
%
