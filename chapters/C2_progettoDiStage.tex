\chapter{Progetto di \emph{stage}}
\label{cap:progettoDiStage}

\section{Visione aziendale}
\subsection{Rapporto dell'azienda con gli \emph{stage}}
Wintech identifica gli \emph{stage} universitari come strumenti utili alla formazione professionale dei laureandi al fine di rendere questi ultimi risorse produttive non solo per il periodo di \emph{stage} definito, ma soprattutto per il periodo successivo. È infatti consuetudine per l'azienda decidere di proseguire i rapporti lavorativi con gli studenti, pertanto, è nel suo interesse rendersi disponibile fornendo approfondimenti e attività formative in ambito tecnologico e riguardo i processi aziendali.\\\\
Per proporre la propria disponibilità e avere un primo contatto con gli studenti, Wintech partecipa da qualche anno all'evento dedicato \gls{STAGEIT} e, successivamente, organizza colloqui conoscitivi al fine di spiegare la propria visione aziendale e i propri progetti di \emph{stage} agli studenti.\\
Tali progetti non sono pensati per essere esclusivamente didattici ma forniscono un tangibile valore all'azienda in quanto si basano sulle tecnologie e metodologie da loro realmente utilizzate o che prevedono di integrare.\\
Lo stagista non è pertanto visto solo come uno studente ma viene integrato al \emph{team} di sviluppo con il quale lavora a stretto contatto acquisendo nuove competenze sia collaborativamente che autonomamente.\\

\subsection{Progetti proposti}
I progetti di \emph{stage} proposti dall'azienda, contestualmente al periodo della nostra collaborazione, derivano dal valore aggiunto che l'automazione dei processi aziendali può portare in ambiti di pianificazione progettuale, per ambienti di sviluppo consolidati e per tecnologie oggetto di ricerca.\\
Esso permette di realizzare efficientemente prodotti con maggiore qualità riducendo sensibilmente i tempi e i costi dei processi.\\
Inoltre, automatizzare attività ben definite riduce sensibilmente la possibilità di introduzione manuale di errori e permette l'individuazione rapida degli stessi.\\
Durante la durata dello \emph{stage}, gli stagisti hanno collaborato più volte trovando dei punti di incontro tra i propri progetti e rimanendo aggiornati riguardo ai rispettivi risultati ottenuti tramite appositi incontri.\\\\
I progetti proposti in questione sono: 

\subsubsection*{Integrazione sistemi di pianificazione di progetto}
Lo scopo dello \emph{stage} è compiere un'analisi per verificare l'utilizzo degli strumenti di pianificazione aziendale, nello specifico Planner e Taiga, e la possibilità di utilizzare le \gls{API} a disposizione al fine di utilizzarle nell'infrastruttura aziendale per sviluppare una soluzione atta ad automatizzare la comunicazione e sincronizzazione di tali strumenti.\\
Le \gls{API} sono degli strumenti che permettono la comunicazione e lo scambio di dati tra diversi componenti \emph{software}.

\subsubsection*{Applicativi di \gls{DevOps} in ambito Sistemi}
\label{stageDavide}
Lo scopo dello \emph{stage} è compiere un'analisi per verificare l'applicabilità delle pratiche di \gls{DevOps} all'ambiente di sviluppo \gls{Sistemi}, con lo scopo di integrarle nell'infrastruttura aziendale in ambiti ben definiti.\\
Le soluzioni individuate durante la ricerca vengono introdotte nel contesto aziendale tramite fasi di sviluppo. 
 
\subsubsection*{Applicativi di \gls{DevOps} in ambito Sistemi e Office365}
\label{mioStage}
Lo scopo dello \emph{stage} da me svolto è compiere un'analisi per verificare l'applicabilità delle pratiche di \gls{DevOps} su progetti basati sull'utilizzo di tecnologie Office365, più specificatamente ai programmi Power Automate e Power Apps, con lo scopo di integrarle nell'infrastruttura aziendale.\\
Le soluzioni individuate durante la ricerca vengono introdotte nel contesto aziendale tramite fasi di sviluppo sia realizzando prototipi dimostrativi (\gls{PoC}), sia implementando tali risultati allo sviluppo di reali applicazioni aziendali realizzate collaborando con il \emph{team}. 

\section{\emph{Stage} da me svolto}
\hyperref[mioStage]{Applicativi di DevOps in ambito Sistemi e Office365}
\subsection{Obiettivi progettuali}
Gli obiettivi dello \emph{stage}, come dichiarati nel documento “Progetto Formativo” generato all'inizio del suo svolgimento, sono divisi in categorie:
\begin{enumerate}
	\item[O -]requisiti obbligatori, vincolanti in quanto obiettivo primario richiesto dall'azienda.
    \item[D -]requisiti desiderabili, non strettamente necessari ma dal riconoscibile valore aggiunto.
    \item[F -]requisiti facoltativi / opzionali, rappresentanti un valore aggiunto non strettamente competitivo.\\\\
\end{enumerate}
Essi sono:
\subsubsection*{Obbligatori}
O1= Mappatura di funzionalità possibili tramite i due applicativi \gls{Sistemi} e Office365:
\begin{itemize}
    \item Analisi approfondita del sistema e delle parti interessate 
    \item Studio delle modalità di lavoro degli utenti 
    \item Produzione di una completa documentazione di uso\\\\
\end{itemize}
O2= Personalizzazione e integrazione:
\begin{itemize}
    \item Individuare modalità di utilizzo\\\\
\end{itemize}
\subsubsection*{Desiderabili}
D1= Analisi dei requisiti per l'integrazione aziendale del \gls{DevOps} in ambito \GLS{Sistemi} e Office365.\\\\
D2= Produzione di una completa documentazione progettuale.\\
\subsubsection*{Facoltativi}
F1= \glslink{PoC}{Proof of concept} contenente alcuni requisiti.\\
F2= Presentazione interna.\\
F3= Predisporre la documentazione.\\\\\\
Tali obiettivi si traducono nelle seguenti aspettative a fine progetto: 
\subsubsection*{Obbligatori}
\begin{itemize}
    \item Documentazione che determini se il \gls{DevOps} aziendale sia applicabile in ambito \gls{Sistemi} eOffice365
    \item Configurazioni e \emph{software} realizzati per determinare l'applicazione del \gls{DevOps} aziendale in ambito \gls{Sistemi} e Office365
\end{itemize}
\subsubsection*{Desiderabili}
\begin{itemize}
    \item Documentazione di sviluppo sicuro annotata con l'esperienza in ambito \gls{Sistemi} eOffice365
\end{itemize}
\subsubsection*{Facoltativi}
\begin{itemize}
    \item Presentazione interna agli \emph{stakeholder} aziendali
    \item Predisposizione di un sistema di esempi e documentazione come guida ai \emph{team}
\end{itemize}


\subsection{Vincoli progettuali}
Lo \emph{stage} ha avuto inizio il giorno 11 settembre ed è terminato il giorno 20 novembre.\\
Esso ha avuto luogo interamente in presenza nella sede di Padova, nella quale mi sono recato dalle ore 9:00 alle ore 13:00 e dalle ore 14:00 alle ore 18:00, per un totale di 320 ore lavorative.\\\\
Tale periodo è stato suddiviso in otto fasi pianificate ben definite, come dichiarato nel documento “Progetto Formativo” generato all'inizio dello stage:
\subsubsection*{Fase 1: dal 11/09/2024 al 13/09/2024 (24h)}
Lo studente insieme agli \emph{stakeholder} prende confidenza con l'ambito di riferimento per lo \emph{stage} e visiona come il ciclo di vita del \emph{software} e il \gls{DevOps} sia stato sponsorizzato in azienda con lo sviluppo sicuro.\\
Lo studente analizza le funzionalità di versionamento con Git e Git Server.\\

\subsubsection*{Fase 2: dal 16/09/2024 al 20/09/2024 (40h) }
Lo studente utilizza le funzionalità di versionamento con Git e il Git Server fornito dall'azienda per versionare il modulo di riferimento, eseguire delle modifiche per dimostrare: salvataggio, modifiche, autori e date, \emph{blame}, \emph{branch}, \emph{merge}, \emph{tag}:\\
Lo studente determina insieme ai \emph{developer} dell'ambito di riferimento se è possibile versionare agevolmente con gli strumenti a disposizione e che differenze ci sono nell'ambito specifico preso a riferimento: \gls{Sistemi}.\\
Lo studente compila con i \emph{developer} il documento di \gls{DevOps} di sviluppo sicuro.\\

\subsubsection*{Fase 3: dal 07/10/2024 al 11/10/2024 (40h) }
Lo studente fornisce i suoi \emph{feedback} su quanto appreso e sponsorizza la sua esperienza ai \emph{team} realizzando documentazione e presentazione.\\

\subsubsection*{Fase 4: dal 14/10/2024 al 18/10/2024 (40h) }
Lo studente valuta se sia possibile utilizzare ambienti di sviluppo \emph{software} (\gls{IDE}) non proprietari in ambito \gls{Sistemi} affiancato dai \emph{developer} \gls{Sistemi}. Lo studente valuta se sia possibile utilizzare SonarLint negli \gls{IDE} in ambito \gls{Sistemi} o altri strumenti di analisi statica utilizzando come guida i documenti di sviluppo sicuro. Lo studente riporta le considerazioni in documentazione ed esegue una presentazione agli \emph{stakeholder}.\\ 

\subsubsection*{Fase 5: dal 21/10/2024 al 25/10/2024 (40h) }
Lo studente discutere con gli \emph{stakeholder} come viene utilizzato Jenkins e come è stato sponsorizzato e analizzato nei documenti di sviluppo sicuro, prendere confidenza con Jenkins realizzando un \emph{job} che compila un modulo di esempio. Lo studente determina se sia stato possibile compilare in ambito \gls{Sistemi} con l'\emph{automation server} Jenkins oppure no come descritto nei documenti di sviluppo sicuro.\\

\subsubsection*{Fase 6: dal 28/10/2024 al 31/10/2024 e dal 04/11/2024 al 08/11/2024 (72h)}
Lo studente insieme ai colleghi inserisce dei \emph{test} di esempio nel modulo e verifica di poterli eseguire con Jenkins e visionare i loro \emph{report}.\\
Verifica le possibilità di \emph{deploy} tramite Jenkins. Verifica come sia composto il \emph{server} in cui eseguire il \emph{deploy} schematizzando le sue parti insieme ai colleghi.\\

\subsubsection*{Fase 7: dal 11/11/2024 al 15/11/2024 (40h) }
Lo studente fornisce i suoi \emph{feedback} finali su quanto appreso tramite l'\emph{automation server} e sponsorizza la sua esperienza ai \emph{team} realizzando documentazione e presentazione.\\

\subsubsection*{Fase 8: dal 18/11/2024 al 20/11/2024 (24h) }
Lo studente interpreta l'esperienza per discutere di vantaggi nell'uso di un sistema di \emph{build} e \emph{deploy} automatizzato e sua sponsorizzazione in un'azienda. Ogni esperienza fatta durate lo \emph{stage} è corredata da documentazione, versionamento di configurazioni o \emph{software} se utilizzati, condivisione con gli \emph{stakeholder}.\\
Viene eseguita una presentazione finale.\\\\\\
Tali indicazioni sono state intese come delle linee guida e non come dei vincoli ferrei, pertanto, ho avuto la libertà di autogestire le mie attività.\\
Esse hanno rispettato quanto descritto con la differenza che è stata data maggiore importanza all'applicazione delle fasi di \gls{DevOps} in ambienti Office365 e allo sviluppo con le tecnologie Power Automate e Power Apps.\\
Inizialmente era incluso fortemente l'ambiente \gls{Sistemi} nelle fasi del progetto poiché era prevista una maggiore collaborazione con lo stage \hyperref[stageDavide]{Applicativi di \gls{DevOps} in ambito Sistemi}.\\
Questi riallineamenti sono avvenuto in maniera naturale e coerentemente con le necessità e indicazioni fornitemi dal tutor aziendale.\\\\
Durante l'avanzamento dei lavori, le attività da me svolte sono sempre state dichiarate tramite lo strumento Planner e tutti i documenti da me realizzati sono stati condivisi in un ambiente comune al fine di mantenere chiarezza con il \emph{team} e il \emph{tutor} aziendale.\\
Con quest'ultimo ho potuto mantenere uno stretto contatto durante il periodo di \emph{stage} in modo da favorire l'interazione e garantire il raggiungimento degli obiettivi prefissati.\\
Ad ogni significativo risultato ottenuto ho realizzato una presentazione successivamente esposta al \emph{tutor} e al \emph{team} di sviluppo.\\

\subsection{Obiettivi personali}
%

