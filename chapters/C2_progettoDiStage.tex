\chapter{Progetto di \emph{stage}}
\label{cap:progettoDiStage}

\section{Visione aziendale}
\subsection{Rapporto dell'azienda con gli \emph{stage}}
Wintech identifica gli \emph{stage} universitari come strumenti utili alla formazione professionale dei laureandi al fine di rendere questi ultimi risorse produttive non solo per il periodo di \emph{stage} definito, ma soprattutto per il periodo successivo. È infatti consuetudine per l'azienda decidere di proseguire i rapporti lavorativi con gli studenti, pertanto, è nel suo interesse rendersi disponibile fornendo approfondimenti e attività formative in ambito tecnologico e riguardo i processi aziendali.\\\\
Per proporre la propria disponibilità e avere un primo contatto con gli studenti, Wintech partecipa da qualche anno all'evento \gls{STAGEIT} e successivamente organizza colloqui conoscitivi al fine di spiegare la propria visione aziendale e i propri progetti di \emph{stage} agli studenti.\\
Tali progetti non sono pensati per essere esclusivamente didattici ma forniscono un tangibile valore all'azienda in quanto si basano sulle tecnologie e metodologie da loro realmente utilizzate o che prevedono di integrare.\\
Lo stagista non è pertanto visto solo come uno studente ma viene integrato al \emph{team} di sviluppo con il quale lavora a stretto contatto acquisendo nuove competenze sia collaborativamente che autonomamente.\\

\subsection{Progetti proposti}
I progetti di \emph{stage} proposti dall'azienda, contestualmente al periodo della nostra collaborazione, derivano dal valore aggiunto che l'automazione dei processi aziendali può portare in ambiti di pianificazione progettuale, per ambienti di sviluppo consolidati e per tecnologie oggetto di ricerca.\\
Esso permette di realizzare efficientemente prodotti con maggiore qualità riducendo sensibilmente i tempi e i costi dei processi.\\
Inoltre, automatizzare attività ben definite riduce sensibilmente la possibilità di introduzione manuale di errori e permette l'individuazione rapida degli stessi.\\
Durante la durata dello \emph{stage}, gli stagisti hanno collaborato più volte trovando dei punti di incontro tra i propri progetti e rimanendo aggiornati riguardo ai rispettivi risultati ottenuti tramite appositi incontri.\\\\
I progetti proposti in questione sono: 

\subsubsection*{Integrazione sistemi di pianificazione di progetto}
Lo scopo dello \emph{stage} è compiere un'analisi per verificare l'utilizzo degli strumenti di pianificazione aziendale, nello specifico Planner e Taiga, e la possibilità di utilizzare le \gls{API} a disposizione al fine di utilizzarle nell'infrastruttura aziendale per sviluppare una soluzione atta ad automatizzare la comunicazione e sincronizzazione di tali strumenti.

\subsubsection*{Applicativi di \gls{DevOps} in ambito Sistemi}
Lo scopo dello \emph{stage} è compiere un'analisi per verificare l'applicabilità delle pratiche di \gls{DevOps} all'ambiente di sviluppo \gls{Sistemi}, con lo scopo di integrarle nell'infrastruttura aziendale in ambiti ben definiti.\\
Le soluzioni individuate durante la ricerca vengono introdotte nel contesto aziendale tramite fasi di sviluppo. 
 
\subsubsection*{Applicativi di \gls{DevOps} in ambito Sistemi e Office365}
\label{mioStage}
Lo scopo dello \emph{stage} da me svolto è compiere un'analisi per verificare l'applicabilità delle pratiche di \gls{DevOps} su progetti basati sull'utilizzo di tecnologie Office365, più specificatamente ai programmi Power Automate e Power Apps, con lo scopo di integrarle nell'infrastruttura aziendale.\\
Le soluzioni individuate durante la ricerca vengono introdotte nel contesto aziendale tramite fasi di sviluppo sia realizzando \gls{PoC} dimostrativi, sia implementando tali risultati allo sviluppo di reali applicazioni aziendali realizzate collaborando con il \emph{team}. 

\section{\emph{Stage} da me svolto}
\hyperref[mioStage]{Applicativi di DevOps in ambito Sistemi e Office365}
\subsection{Obiettivi progettuali}
Gli obiettivi dello \emph{stage}, come dichiarati nel documento “Progetto Formativo” generato all'inizio del suo svolgimento, sono divisi in categorie:
\begin{enumerate}
	\item[O -]requisiti obbligatori, vincolanti in quanto obiettivo primario richiesto dall'azienda.
    \item[D -]requisiti desiderabili, non strettamente necessari ma dal riconoscibile valore aggiunto.
    \item[F -]requisiti facoltativi / opzionali, rappresentanti un valore aggiunto non strettamente competitivo.\\\\
\end{enumerate}
Essi sono:
\subsubsection*{Obbligatori}
O1= Mappatura di funzionalità possibili tramite i due applicativi \gls{Sistemi} e Office365:
\begin{itemize}
    \item Analisi approfondita del sistema e delle parti interessate 
    \item Studio delle modalità di lavoro degli utenti 
    \item Produzione di una completa documentazione di uso\\\\
\end{itemize}
O2= Personalizzazione e integrazione:
\begin{itemize}
    \item Individuare modalità di utilizzo\\\\
\end{itemize}
\subsubsection*{Desiderabili}
D1= Analisi dei requisiti per l'integrazione aziendale del \gls{DevOps} in ambito \GLS{Sistemi} e Office365.\\\\
D2= Produzione di una completa documentazione progettuale.\\
\subsubsection*{Facoltativi}
F1= \glslink{PoC}{Proof of concept} contenente alcuni requisiti.\\
F2= Presentazione interna.\\
F3= Predisporre la documentazione.\\\\\\
Tali obiettivi si traducono nelle seguenti aspettative a fine progetto: 
\subsubsection*{Obbligatori}
\begin{itemize}
    \item Documentazione che determini se il \gls{DevOps} aziendale sia applicabile in ambito \gls{Sistemi} eOffice365
    \item Configurazioni e \emph{software} realizzati per determinare l'applicazione del \gls{DevOps} aziendale in ambito \gls{Sistemi} e Office365
\end{itemize}
\subsubsection*{Desiderabili}
\begin{itemize}
    \item Documentazione di sviluppo sicuro annotata con l'esperienza in ambito \gls{Sistemi} eOffice365
\end{itemize}
\subsubsection*{Facoltativi}
\begin{itemize}
    \item Presentazione interna agli \emph{stakeholder} aziendali
    \item Predisposizione di un sistema di esempi e documentazione come guida ai \emph{team}
\end{itemize}


\subsection{Vincoli progettuali}


\subsection{Obiettivi personali}
%

