% \omiss produces '[...]'
\newcommand{\omissis}{[\dots\negthinspace]}

% Itemize symbols
% see: https://tex.stackexchange.com/a/62497
% \renewcommand{\labelitemi}{$\bullet$}
% \renewcommand{\labelitemii}{$\cdot$}
% \renewcommand{\labelitemiii}{$\diamond$}
% \renewcommand{\labelitemiv}{$\ast$}


\let\Chaptermark\chaptermark
% \Chaptername gives current chapter name
\def\chaptermark#1{\def\Chaptername{#1}\Chaptermark{#1}}
\makeatletter
% \currentname gives the current section name
\newcommand*{\currentname}{\@currentlabelname}
\makeatother

% Uncomment the following line for a different header/footer style
% \pagestyle{fancy} \setlength{\headheight}{14.5pt}
\fancyhead[L]{\fontsize{12}{14.5} \selectfont \thechapter. \Chaptername}
\fancyhead[R]{\fontsize{12}{14.5} \selectfont \currentname}
% Page number always in footer
\cfoot{\thepage}


% Custom hyphenation rules
\hyphenation {
    e-sem-pio
    ex-am-ple
}

% Images path, not using \graphicspath because it doesn't properly work with
% latexmk custom dependencies
\NewCommandCopy{\latexincludegraphics}{\includegraphics}
\renewcommand{\includegraphics}[2][]{\latexincludegraphics[#1]{../images/#2}}

% Page format settings
% see: http://wwwcdf.pd.infn.it/AppuntiLinux/a2547.htm
\setlength{\parindent}{14pt}    % first row indentation
\setlength{\parskip}{0pt}       % paragraphs spacing


% Load variables
\newcommand{\myName}{Nome Cognome}
\newcommand{\myID}{1234256}
\newcommand{\myTitle}{Titolo della tesi tmp}
\newcommand{\myDegree}{Tesi di laurea}
\newcommand{\myUni}{Università degli Studi di Padova}
% For BSc level just use "Corso di Laurea" and don't add "Triennale" to it
\newcommand{\myFaculty}{Corso di Laurea in Informatica}
\newcommand{\myDepartment}{Dipartimento di Matematica ``Tullio Levi-Civita''}
\newcommand{\profTitle}{Prof.}
\newcommand{\myProf}{Nome Cognome}
\newcommand{\myLocation}{Padova}
\newcommand{\myAA}{AAAA-AAAA}
\newcommand{\myTime}{Mese AAAA}

% PDF file metadata fields
% when updating them delete the build directory, otherwise they won't change
\begin{filecontents*}{\jobname.xmpdata}
  \Title{Document's title}
  \Author{Author's name}
  \Language{it-IT}
  \Subject{Short description}
  \Keywords{keyword1\sep keyword2\sep keyword3}
\end{filecontents*}


\makeglossaries
% Acronyms
\newacronym[description={\glslink{apig}{Application Program Interface}}]
    {api}{API}{Application Program Interface}

\newacronym[description={\glslink{umlg}{Unified Modeling Language}}]
    {uml}{UML}{Unified Modeling Language}

% Glossary entries
\newglossaryentry{system integration}{
    name={System integration},
    text=system integration,
    description={Processo di combinazione di diversi componenti \emph{software} e infrastrutture in un sistema unico e coeso, al fine di garantire che le parti lavorino insieme in modo efficiente e sinergico. Questo processo include la connessione di sistemi esistenti con nuove tecnologie per migliorare le funzionalità, la condivisione dei dati e il coordinamento delle operazioni}
}

\newglossaryentry{IT}{
    name={Information Technology (IT)},
    text=IT,
    description={Termine che si riferisce all'uso di tecnologie, dispositivi e sistemi per creare, memorizzare, elaborare, scambiare e proteggere informazioni e dati. Include infrastrutture \emph{hardware}, \emph{software}, reti e servizi correlati}
}

\newglossaryentry{LAN}{
    name={Local Area Network (LAN)},
    text=LAN,
    description={Rete informatica locale che connette dispositivi, come \emph{computer}, stampanti e \emph{server}, all'interno di un'area limitata, ad esempio un edificio, un ufficio o una scuola. Le LAN consentono la condivisione di risorse e la comunicazione tra dispositivi con velocità elevate e basse latenze}
}

\newglossaryentry{WAN}{
    name={Wide Area Network (WAN)},
    text=WAN,
    description={Rete di comunicazione che collega dispositivi o reti locali (\gls{LAN}) su una vasta area geografica, come città, nazioni o continenti. Le WAN utilizzano infrastrutture pubbliche o private per trasmettere dati su lunghe distanze, consentendo la condivisione di informazioni tra utenti e sistemi remoti}
}

\newglossaryentry{ITIL}{
    name={Information Technology Infrastructure Library (ITIL)},
    text=ITIL,
    description={Insieme di linee guida per la gestione dei servizi \gls{IT} al fine di migliorarne l'erogazione, il supporto e la qualità, mantenendo un allineamento con gli obbiettivi aziendali}
}

\newglossaryentry{DevOps}{
    name={DevOps},
    description={Cultura, metodologia e insieme di pratiche che uniscono sviluppo \emph{software} (Dev) e operazioni \gls{IT} (Ops) per migliorare la collaborazione, l'efficienza e la velocità nella creazione, distribuzione e gestione delle applicazioni, mantenendo alta la qualità e la stabilità dei servizi. DevOps enfatizza l'automazione, la condivisione di responsabilità e il miglioramento continuo, utilizzando strumenti e processi che supportano la \gls{Continuous Integration}, il \gls{Continuous Deployment} e il monitoraggio costante dei sistemi}
}

\newglossaryentry{disaster recovery}{
    name={Disaster recovery},
    text=disaster recovery,
    description={Processo e insieme di strategie volte a ripristinare sistemi, dati e infrastrutture \gls{IT} critiche dopo un evento catastrofico, come guasti \emph{hardware}, attacchi informatici, disastri naturali o errori umani}
}

\newglossaryentry{ERP}{
    name={Enterprise Resource Planning (ERP)},
    text=ERP,
    description={Sistema \emph{software} integrato utilizzato per gestire e ottimizzare i processi aziendali fondamentali come contabilità, gestione delle risorse umane, produzione, vendite e \emph{marketing}. Gli ERP centralizzano i dati in un'unica piattaforma, migliorando la condivisione delle informazioni, l'efficienza operativa e il processo decisionale}
}

\newglossaryentry{PMI}{
    name={Piccole e Medie Imprese (PMI)},
    text= PMI,
    description={Categoria di aziende che, in base a dimensioni e fatturato, rientrano nelle definizioni stabilite da enti nazionali o internazionali. Nell'Unione europea esse sono contraddistinte da un numero di dipendenti inferiore a 250 e un fatturato inferiore o uguale a 50 milioni di euro}
}

\newglossaryentry{cloud}{
    name={Cloud},
    text=cloud,
    description={In ambito informatico, il \emph{cloud} si riferisce all'uso di risorse \gls{IT} (come \emph{server}, \emph{storage}, \emph{database}, \emph{software} e reti) attraverso Internet, anziché dipendere da infrastrutture fisiche locali. I servizi \emph{cloud} permettono alle aziende e agli utenti di archiviare e accedere ai dati e alle applicazioni da qualsiasi luogo, con vantaggi in termini di scalabilità, flessibilità, costi ridotti e aggiornamenti automatici}
}

\newglossaryentry{back-end}{
    name={Back-end},
    text=cloud,
    description={Parte di un'applicazione o di un sistema informatico che gestisce la logica, il \emph{database}, la sicurezza e la gestione dei dati, ed è responsabile della comunicazione con l'interfaccia utente. Gli sviluppatori \emph{back-end} si occupano di scrivere il codice che permette di gestire le operazioni come la gestione degli utenti, l'elaborazione dei dati e la gestione delle richieste provenienti dall'interfaccia grafica}
}

\newglossaryentry{project manager}{
    name={Project Manager (PM)},
    text=project manager,
    description={Professionista responsabile della pianificazione, esecuzione e supervisione di un progetto. Egli gestisce le risorse, coordina il \emph{team}, monitora i progressi, risolve i problemi e comunica con clienti e fornitori per garantire il successo del progetto assicurandosi che gli obiettivi vengano raggiunti nei tempi, nel \emph{budget} e con la qualità prevista}
}

\newglossaryentry{unità operativa}{
    name={Unità operativa},
    text=unità operativa,
    description={Anche denominata \emph{Business unit} (o BU), è una struttura o divisione all'interno di un'organizzazione che si occupa di specifiche attività operative per il raggiungimento degli obiettivi aziendali. Ogni unità operativa è generalmente responsabile di un'area funzionale specifica, come la produzione, la logistica, la vendita, o il servizio clienti}
}

\newglossaryentry{caso d'uso}{
    name={Caso d'uso},
    text=caso d'uso,
    description={In inglese \emph{Use Case} (o UC), è utilizzato principalmente nell'ambito dell'analisi e della progettazione dei sistemi \emph{software} e rappresenta la descrizione di una sequenza di azioni che il sistema esegue in risposta a una richiesta dell'utente. Sono incluse le varie possibilità di esito (ad esempio, scenari di successo o di errore), al fine di definire il suo comportamento in relazione ai requisiti prestabiliti}
}

\newglossaryentry{Continuous Integration}{
    name={Continuous Integration (CI)},
    text=Continuous Integration,
    description={Pratica di sviluppo \emph{software} in cui i membri di un \emph{team} integrano frequentemente il loro lavoro in un \emph{repository} comune, generalmente più volte al giorno. Ogni integrazione viene verificata automaticamente attraverso l'esecuzione di \emph{test} per rilevare rapidamente eventuali errori o conflitti nel codice. L'obiettivo della \emph{Continuous Integration} è ridurre i problemi di integrazione, migliorare la qualità del \emph{software} e accelerare il ciclo di sviluppo, permettendo modifiche più rapide e sicure}
}

\newglossaryentry{Continuous Deployment}{
    name={Continuous Deployment (CD)},
    text=Continuous Deployment,
    description={Pratica di sviluppo \emph{software} in cui ogni modifica al codice che supera i \emph{test} automatici viene automaticamente distribuita in produzione senza intervento manuale. Questo approccio consente di rilasciare nuove versioni del \emph{software} in modo rapido e frequente, garantendo che le funzionalità siano disponibili per gli utenti finali in tempi brevi. L'automazione completa del processo riduce i rischi di errore umano e migliora la velocità e l'affidabilità dei rilasci}
}



\bibliography{appendix/bibliography}

\defbibheading{bibliography} {
    \cleardoublepage
    \phantomsection
    \addcontentsline{toc}{chapter}{\bibname}
    \chapter*{\bibname\markboth{\bibname}{\bibname}}
}

% Spacing between entries
\setlength\bibitemsep{1.5\itemsep}

\DeclareBibliographyCategory{opere}
\DeclareBibliographyCategory{web}

\addtocategory{opere}{womak:lean-thinking}
\addtocategory{web}{site:agile-manifesto}

\defbibheading{opere}{\section*{Riferimenti bibliografici}}
\defbibheading{web}{\section*{Siti Web consultati}}


\captionsetup{
    tableposition=top,
    figureposition=bottom,
    font=small,
    format=hang,
    labelfont=bf
}

\hypersetup{
    %hyperfootnotes=false,
    %pdfpagelabels,
    colorlinks=true,
    linktocpage=true,
    pdfstartpage=1,
    pdfstartview=,
    breaklinks=true,
    pdfpagemode=UseNone,
    pageanchor=true,
    pdfpagemode=UseOutlines,
    plainpages=false,
    bookmarksnumbered,
    bookmarksopen=true,
    bookmarksopenlevel=1,
    hypertexnames=true,
    pdfhighlight=/O,
    %nesting=true,
    %frenchlinks,
    urlcolor=webbrown,
    linkcolor=RoyalBlue,
    citecolor=webgreen
    %pagecolor=RoyalBlue,
}

% Delete all links and show them in black
\if \isprintable 1
    \hypersetup{draft}
\fi

% Listings setup
\lstset{
    language=[LaTeX]Tex,%C++,
    keywordstyle=\color{RoyalBlue}, %\bfseries,
    basicstyle=\small\ttfamily,
    %identifierstyle=\color{NavyBlue},
    commentstyle=\color{Green}\ttfamily,
    stringstyle=\rmfamily,
    numbers=none, %left,%
    numberstyle=\scriptsize, %\tiny
    stepnumber=5,
    numbersep=8pt,
    showstringspaces=false,
    breaklines=true,
    frameround=ftff,
    frame=single
}

\definecolor{webgreen}{rgb}{0,.5,0}
\definecolor{webbrown}{rgb}{.6,0,0}

\newcommand{\sectionname}{sezione}
\addto\captionsitalian{\renewcommand{\figurename}{Figura}
                       \renewcommand{\tablename}{Tabella}}

\newcommand{\glsfirstoccur}{\ap{{[g]}}}

\newcommand{\intro}[1]{\emph{\textsf{#1}}}

% Risks environment
\newcounter{riskcounter}                % define a counter
\setcounter{riskcounter}{0}             % set the counter to some initial value

%%%% Parameters
% #1: Title
\newenvironment{risk}[1]{
    \refstepcounter{riskcounter}        % increment counter
    \par \noindent                      % start new paragraph
    \textbf{\arabic{riskcounter}. #1}   % display the title before the content of the environment is displayed
}{
    \par\medskip
}

\newcommand{\riskname}{Rischio}

\newcommand{\riskdescription}[1]{\textbf{\\Descrizione:} #1.}

\newcommand{\risksolution}[1]{\textbf{\\Soluzione:} #1.}

% Use case environment
\newcounter{usecasecounter}             % define a counter
\setcounter{usecasecounter}{0}          % set the counter to some initial value

%%%% Parameters
% #1: ID
% #2: Nome
\newenvironment{usecase}[2]{
    \renewcommand{\theusecasecounter}{\usecasename #1}  % this is where the display of
                                                        % the counter is overwritten/modified
    \refstepcounter{usecasecounter}             % increment counter
    \vspace{10pt}
    \par \noindent                              % start new paragraph
    {\large \textbf{\usecasename #1: #2}}       % display the title before the
                                                % content of the environment is displayed
    \medskip
}{
    \medskip
}

\newcommand{\usecasename}{UC}

\newcommand{\usecaseactors}[1]{\textbf{\\Attori Principali:} #1. \vspace{4pt}}
\newcommand{\usecasepre}[1]{\textbf{\\Precondizioni:} #1. \vspace{4pt}}
\newcommand{\usecasedesc}[1]{\textbf{\\Descrizione:} #1. \vspace{4pt}}
\newcommand{\usecasepost}[1]{\textbf{\\Postcondizioni:} #1. \vspace{4pt}}
\newcommand{\usecasealt}[1]{\textbf{\\Scenario Alternativo:} #1. \vspace{4pt}}

% Namespace description environment
\newenvironment{namespacedesc}{
    \vspace{10pt}
    \par \noindent  % start new paragraph
    \begin{description}
}{
    \end{description}
    \medskip
}

\newcommand{\classdesc}[2]{\item[\textbf{#1:}] #2}
