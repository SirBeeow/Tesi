% Acronyms
\newacronym[description={\glslink{apig}{Application Program Interface}}]
    {api}{API}{Application Program Interface}

\newacronym[description={\glslink{umlg}{Unified Modeling Language}}]
    {uml}{UML}{Unified Modeling Language}

% Glossary entries
\newglossaryentry{IT}{
    name={Information Technology (IT)},
    text=IT,
    description={Sottocategoria dell'\gls{ICT}, include infrastrutture \emph{hardware}, \emph{software}, reti e servizi correlati. Numerose industrie sono legate alla tecnologia dell'informazione, inclusi produttori di elettronica, semiconduttori, attrezzature per la telecomunicazione, commercio elettronico, \emph{web design} e servizi informatici. L'infrastruttura IT può esistere localmente oppure in un ambiente \gls{cloud}}
}

\newglossaryentry{DevOps}{
    name={DevOps},
    description={Metodologia che enfatizza l'automazione, la condivisione di responsabilità e il miglioramento continuo, utilizzando strumenti e processi che supportano la \gls{Continuous Integration}, il \gls{Continuous Deployment} e il monitoraggio costante dei sistemi}
}

\newglossaryentry{PMI}{
    name={Piccole e Medie Imprese (PMI)},
    text= PMI,
    description={Categoria di aziende che, in base a dimensioni e fatturato, rientrano nelle definizioni stabilite da enti nazionali o internazionali. Nell'Unione europea esse sono contraddistinte da un numero di dipendenti inferiore a 250 e un fatturato inferiore o uguale a 50 milioni di euro}
}

\newglossaryentry{cloud}{
    name={Cloud},
    text=cloud,
    description={I servizi \emph{cloud} permettono alle aziende e agli utenti di archiviare e accedere ai dati e alle applicazioni da qualsiasi luogo, con vantaggi in termini di scalabilità, flessibilità, costi ridotti e aggiornamenti automatici}
}

\newglossaryentry{back-end}{
    name={Back-end},
    text=back-end,
    description={Gli sviluppatori \emph{back-end} si occupano di scrivere il codice che permette di gestire le operazioni come la gestione degli utenti, l'elaborazione dei dati e la gestione delle richieste provenienti dall'interfaccia grafica. Alcuni esempi di linguaggi di programmazione utilizzati per la scrittura della logica lato \emph{server} sono Ruby, Java e Python}
}

\newglossaryentry{Continuous Integration}{
    name={Continuous Integration (CI)},
    text=Continuous Integration,
    description={Ogni integrazione viene verificata automaticamente attraverso l'esecuzione di \emph{test} per rilevare rapidamente eventuali errori o conflitti nel codice. Il concetto della \emph{Continuous Integration} è stato originariamente proposto come contromisura preventiva per il problema dell'"\emph{integration hell}", ovvero le difficoltà dell'integrazione di porzioni di \emph{software} sviluppati in modo indipendente su lunghi periodi di tempo e che di conseguenza potrebbero essere significativamente divergenti}
}

\newglossaryentry{Continuous Deployment}{
    name={Continuous Deployment (CD)},
    text=Continuous Deployment,
    description={L'adozione di questo approccio consente di rilasciare nuove versioni del \emph{software} in modo rapido e frequente, garantendo che le funzionalità siano disponibili per gli utenti finali in tempi brevi. Inoltre, il \emph{team} di sviluppo non è più obbligato ad interrompere lo sviluppo per prepararsi ed effettuare i rilasci. Questi ultimi sono meno rischiosi poiché le modifiche apportate al prodotto sono tipicamente contenute ed è quindi più agevole identificare eventuali problemi. Infine il cliente ha la possibilità di fornire \emph{feedback} costantemente potendo verificare ogni avanzamento}
}

\newglossaryentry{ICT}{
    name={Tecnologie dell'Informazione e della Comunicazione (ICT)},
    text=ICT,
    description={Il termine include una vasta gamma di strumenti e risorse, come \emph{computer}, \emph{software}, reti di telecomunicazione, \emph{internet} e dispositivi mobili, che consentono la creazione, l'archiviazione, la gestione e lo scambio di dati e contenuti}
}

\newglossaryentry{Sistemi}{
    name={Sistemi},
    description={Sistemi S.p.A. è una società italiana partecipata con Wintech S.p.A.. Essa possiede tecnologie ed ambienti di sviluppo dedicati al fine di creare soluzioni \emph{software} gestionali e servizi per professionisti e imprese, soprattutto in ambiti relativi a studi professionali di commercialisti, consulenti del lavoro e avvocati, imprese e associazioni di categoria}
}

\newglossaryentry{http}{
    name={HyperText Transfer Protocol (HTTP)},
    text=HTTP,
    description={Protocollo a livello applicativo, ovvero il livello più alto definito dal modello OSI (Open Systems Interconnection), il quale rappresenta uno \emph{standard} architetturale per reti di calcolatori. Tale livello è responsabile della gestione delle comunicazioni tra applicazioni, fornendo i servizi necessari per lo scambio di dati strutturati e significativi tra \emph{client} e \emph{server}}
}
