% Acronyms
\newacronym[description={\glslink{apig}{Application Program Interface}}]
    {api}{API}{Application Program Interface}

\newacronym[description={\glslink{umlg}{Unified Modeling Language}}]
    {uml}{UML}{Unified Modeling Language}

% Glossary entries
\newglossaryentry{system integration}{
    name={System integration},
    text=system integration,
    description={Processo di combinazione di diversi componenti \emph{software} e infrastrutture in un sistema unico e coeso, al fine di garantire che le parti lavorino insieme in modo efficiente e sinergico. Questo processo include la connessione di sistemi esistenti con nuove tecnologie per migliorare le funzionalità, la condivisione dei dati e il coordinamento delle operazioni}
}

\newglossaryentry{IT}{
    name={Information Technology (IT)},
    text=IT,
    description={Termine che si riferisce all'uso di tecnologie, dispositivi e sistemi per creare, memorizzare, elaborare, scambiare e proteggere informazioni e dati. Include infrastrutture \emph{hardware}, \emph{software}, reti e servizi correlati}
}

\newglossaryentry{LAN}{
    name={Local Area Network (LAN)},
    text=LAN,
    description={Rete informatica locale che connette dispositivi, come \emph{computer}, stampanti e \emph{server}, all'interno di un'area limitata, ad esempio un edificio, un ufficio o una scuola. Le LAN consentono la condivisione di risorse e la comunicazione tra dispositivi con velocità elevate e basse latenze}
}

\newglossaryentry{WAN}{
    name={Wide Area Network (WAN)},
    text=WAN,
    description={Rete di comunicazione che collega dispositivi o reti locali (\gls{LAN}) su una vasta area geografica, come città, nazioni o continenti. Le WAN utilizzano infrastrutture pubbliche o private per trasmettere dati su lunghe distanze, consentendo la condivisione di informazioni tra utenti e sistemi remoti}
}

\newglossaryentry{ITIL}{
    name={Information Technology Infrastructure Library (ITIL)},
    text=ITIL,
    description={Insieme di linee guida per la gestione dei servizi \gls{IT} al fine di migliorarne l'erogazione, il supporto e la qualità, mantenendo un allineamento con gli obbiettivi aziendali}
}

\newglossaryentry{DevOps}{
    name={DevOps},
    description={Cultura, metodologia e insieme di pratiche che uniscono sviluppo \emph{software} (Dev) e operazioni \gls{IT} (Ops) per migliorare la collaborazione, l'efficienza e la velocità nella creazione, distribuzione e gestione delle applicazioni, mantenendo alta la qualità e la stabilità dei servizi. DevOps enfatizza l'automazione, la condivisione di responsabilità e il miglioramento continuo, utilizzando strumenti e processi che supportano la \gls{Continuous Integration}, il \gls{Continuous Deployment} e il monitoraggio costante dei sistemi}
}

\newglossaryentry{disaster recovery}{
    name={Disaster recovery},
    text=disaster recovery,
    description={Processo e insieme di strategie volte a ripristinare sistemi, dati e infrastrutture \gls{IT} critiche dopo un evento catastrofico, come guasti \emph{hardware}, attacchi informatici, disastri naturali o errori umani}
}

\newglossaryentry{ERP}{
    name={Enterprise Resource Planning (ERP)},
    text=ERP,
    description={Sistema \emph{software} integrato utilizzato per gestire e ottimizzare i processi aziendali fondamentali come contabilità, gestione delle risorse umane, produzione, vendite e \emph{marketing}. Gli ERP centralizzano i dati in un'unica piattaforma, migliorando la condivisione delle informazioni, l'efficienza operativa e il processo decisionale}
}

\newglossaryentry{PMI}{
    name={Piccole e Medie Imprese (PMI)},
    text= PMI,
    description={Categoria di aziende che, in base a dimensioni e fatturato, rientrano nelle definizioni stabilite da enti nazionali o internazionali. Nell'Unione europea esse sono contraddistinte da un numero di dipendenti inferiore a 250 e un fatturato inferiore o uguale a 50 milioni di euro}
}

\newglossaryentry{cloud}{
    name={Cloud},
    text=cloud,
    description={In ambito informatico, il \emph{cloud} si riferisce all'uso di risorse \gls{IT} (come \emph{server}, \emph{storage}, \emph{database}, \emph{software} e reti) attraverso Internet, anziché dipendere da infrastrutture fisiche locali. I servizi \emph{cloud} permettono alle aziende e agli utenti di archiviare e accedere ai dati e alle applicazioni da qualsiasi luogo, con vantaggi in termini di scalabilità, flessibilità, costi ridotti e aggiornamenti automatici}
}

\newglossaryentry{back-end}{
    name={Back-end},
    text=cloud,
    description={Parte di un'applicazione o di un sistema informatico che gestisce la logica, il \emph{database}, la sicurezza e la gestione dei dati, ed è responsabile della comunicazione con l'interfaccia utente. Gli sviluppatori \emph{back-end} si occupano di scrivere il codice che permette di gestire le operazioni come la gestione degli utenti, l'elaborazione dei dati e la gestione delle richieste provenienti dall'interfaccia grafica}
}

\newglossaryentry{project manager}{
    name={Project Manager (PM)},
    text=project manager,
    description={Professionista responsabile della pianificazione, esecuzione e supervisione di un progetto. Egli gestisce le risorse, coordina il \emph{team}, monitora i progressi, risolve i problemi e comunica con clienti e fornitori per garantire il successo del progetto assicurandosi che gli obiettivi vengano raggiunti nei tempi, nel \emph{budget} e con la qualità prevista}
}

\newglossaryentry{unità operativa}{
    name={Unità operativa},
    text=unità operativa,
    description={Anche denominata \emph{Business unit} (o BU), è una struttura o divisione all'interno di un'organizzazione che si occupa di specifiche attività operative per il raggiungimento degli obiettivi aziendali. Ogni unità operativa è generalmente responsabile di un'area funzionale specifica, come la produzione, la logistica, la vendita, o il servizio clienti}
}

\newglossaryentry{caso d'uso}{
    name={Caso d'uso},
    text=caso d'uso,
    description={In inglese \emph{Use Case} (o UC), è utilizzato principalmente nell'ambito dell'analisi e della progettazione dei sistemi \emph{software} e rappresenta la descrizione di una sequenza di azioni che il sistema esegue in risposta a una richiesta dell'utente. Sono incluse le varie possibilità di esito (ad esempio, scenari di successo o di errore), al fine di definire il suo comportamento in relazione ai requisiti prestabiliti}
}

\newglossaryentry{Continuous Integration}{
    name={Continuous Integration (CI)},
    text=Continuous Integration,
    description={Pratica di sviluppo \emph{software} in cui i membri di un \emph{team} integrano frequentemente il loro lavoro in un \emph{repository} comune, generalmente più volte al giorno. Ogni integrazione viene verificata automaticamente attraverso l'esecuzione di \emph{test} per rilevare rapidamente eventuali errori o conflitti nel codice. L'obiettivo della \emph{Continuous Integration} è ridurre i problemi di integrazione, migliorare la qualità del \emph{software} e accelerare il ciclo di sviluppo, permettendo modifiche più rapide e sicure}
}

\newglossaryentry{Continuous Deployment}{
    name={Continuous Deployment (CD)},
    text=Continuous Deployment,
    description={Pratica di sviluppo \emph{software} in cui ogni modifica al codice che supera i \emph{test} automatici viene automaticamente distribuita in produzione senza intervento manuale. Questo approccio consente di rilasciare nuove versioni del \emph{software} in modo rapido e frequente, garantendo che le funzionalità siano disponibili per gli utenti finali in tempi brevi. L'automazione completa del processo riduce i rischi di errore umano e migliora la velocità e l'affidabilità dei rilasci}
}
